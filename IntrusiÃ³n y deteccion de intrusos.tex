\documentclass[11pt,a4paper]{article}

\usepackage[spanish]{}
\usepackage[utf8]{inputenc}
\usepackage{amsmath}
\usepackage{amssymb}
\usepackage{float}
\usepackage{listings}
\usepackage{gensymb}
\usepackage{xparse}% http://ctan.org/pkg/xparse
\usepackage[inline]{enumitem}
\usepackage{hyperref}

\newtheorem{definicion}{Definición}
\newtheorem{nota}{Nota}
\newtheorem{teorema}{Teorema}
\newtheorem{lema}{Lema}
\newtheorem{observacion}{Observación}
\newtheorem{demostracion}{Demostración}
\newtheorem{corolario}{Corolario}
\newtheorem{ejemplo}{Ejemplo}

\begin{document}

\title{Criptografía y seguridad\\Intrusión y detección de intrusos}
\author{Montoya Montes Pedro\\
		Universidad Nacional Autónoma de Mexico}
\maketitle
\tableofcontents

\newpage
\begin{center}
PAGINA INTENCIONALMENTE EN BLANCO
\end{center}

\newpage
\section{Intrusión}
Primeramente vamos a definir que una intrusión es el acceso de forma ilicita a un sistema.\\
Existen dos tipos de intrusión:
\begin{enumerate}
	\item Directa: Es aquella en la que se tiene acceso directo al equipo al que se quiere tener acceso.
	\item Indirecta: Es la que mediante una red, ya sea pública o privada, se tiene acceso a un equipo.
\end{enumerate}
\section{Vulnerabilidad}
Como define Toni Puig,  una  vulnerabilidad “es  la  potencialidad  o posibilidad  de  ocurrencia  de  la  materialización  de una  amenaza  sobre  un activo”. \footnote{ Toni Puig, “Gestión de riesgos de los sistemas de información” en http://www.mailxmail.com/curso-gestion-riesgos-sistemas-informacion/identificacion-vulnerabilidades-impactos, 12/02/201}\\
La podemos dividir en tres grupos:
\begin{itemize}
	\item[•] Vulnerabilidad intrínseca, es aquella que sólo depende del activo y de la amenaza en sí. 
	\item[•] Vulnerabilidad  efectiva, es la resultante luego de que se aplican las medidas correspondientes. 
	\item[•] Vulnerabilidad  residual, es la resultante luego de que se aplican las medidas complementarias.
\end{itemize}
Al mismo tiempo existen tres tipos de intrusos.\footnote{ https://www.solvetic.com/page/recopilaciones/s/profesionales/tipos-de-ataques-informaticos-e-intrusos-y-como-detectarlos}
\begin{itemize}
	\item[•] Usuario fraudulento: Hace referencia a un usuario que accede de manera ilegal a recursos de la organización o que, teniendo los permisos, hace uso indebido de la información disponible. 
	\item[•] Suplantador: Es una persona que no tiene nada que ver con los accesos legales en la organización pero que logra llegar hasta el nivel de tomar la identidad de un usuario legitimo para acceder y realizar el daño.
	\item[•] Usuario clandestino: Es una persona que puede tomar el control de auditoria del sistema de la organización.
\end{itemize}

\section{Ataques}
Existen varios tipos de ataques que ayudan a detectar distintas caracteristicas de la victima y así poder efectuar un ataque de mejor manera, a destacar serian los siguientes:
\subsection{Barrido de puertos}
Esta técnica nos es útil para saber la situación de cierto puerto, es decir, si se encuentra abierto, cerrado o tras un firewall.\\
Existen los siguientes tipos de barrido de puertos:
\subsubsection{Escaneo de puertos TCP}
Cada vez que se establece una conexión TCP (Transmission Control Protocol), se sigue una negociación conocida como Three-way-handshake, que consiste en:\\
La máquina de origen envía un paquete con bandera SYN activa.\\
La máquina destino responde con una bandera SYN/ACK.\\
La máquina origen envía un paquete en la bandera ACK.\\
Una vez finalizado, la conexión se ha completado entre la máquina origen y destino.\\
El escáneo de puertos TCP envía a la victima varios paquetes SYN, y dependiendo la respuesta puede intepretar los siguiente:
\begin{itemize}
	\item[•] Si la respuesta es un SYN/ACK, entonces el puerto está abierto.
	\item[•] Si la respuesta es un paquete RST, significará que el puerto está cerrado.
	\item[•] Si la respuesta es  un  paquete ICMP  (Internet  Control Message  Protocol) como puerto inalcanzable, será porque dicho puerto está protegido por un firewall.
\end{itemize}

\subsubsection{Escaneo 	de puertos UDP}
A pesar de que el protocolo UDP (User Datagram Protocol) no está orientado a conexiones, si se envia un paquete a dicho puerto y se encuentra cerrado, se recibirá un mensaje de puerto inalcanzable, si embargo, si no se obtiene respuesta alguna, se difiere que está abierto, aunque si dicho puerto está protegido por un firewall la respuesta obtenida no será concisa.

\subsection{Identificación de firewalls}
Se podría definir a un firewall como un dispositivo informático, ya sea software ohardware que protege a una red privada al definir un perímetro de seguridad, y definírsele reglas de filtrado de paquetes.  La  función  principal  del firewall  es  la  de  examinar  paquetes  en  busca  de coincidencias  con  las  reglas  que  se  le  han  fijado  y  dependiendo  de  ellas, permitirles o negarles el acceso. Además, los firewalls pueden también generar alarmas y crear listas negras
.\footnote{ \url {http://tindes.com/jj/index.php?option=com_content&view=article&id=41:seguridad-red&catid=1:actualidad&Itemid=27&0de28b3435550b272401162583c2c73f=tqyhgghphlnuox}}
Existen tres tipos:
\subsubsection{Firewall de filtrado por paquetes}
Trabaja en la capa tres del modelo OSI, analizando el encabezado del paquete de procedencia y destino, con estos datos se determina si se permine o niega el paso al paquete.
\subsubsection{Firewall de filtrado por estado}
Este permite guardar un registro de las conexiones existentes. Además, es capaz de  trabajar  con  los  protocolos  tales  como IP, TCP, UDP, ICMP, FTP  e IRC.  No trabajan  filtrando  paquetes  individuales, sino  sesiones  enteras, permitiendo una optimización del trabajo de filtrado. 
\footnote{ \url{ http://www.alegsa.com.ar/Dic/seguridad/$\%$50informatica.php}}
\subsubsection{Firewall de filtrado por contenido}
También conocido como de filtrado por aplicación, analiza la trama a nivel de la capa  de  aplicación  del  modelo OSI,  así,  controla  además  de  los  puertos  y sesiones,  los  protocolos  que  se  utilizan  para  la  comunicación  evitando  que  se puedan suplantar servicios.

\subsection{Ientificación del sistema operativo}
Para está técnica, que se utiliza para saber el sistema operativo donde se corre el sistema a atacar, y se tienen dos variantes activa y pásiva.\\
La primera se dedica a enviar y recibir paquetes para posteriormente analizar y saber el sistema operativo.\\
Mientras que la segunda, se dedica a recabar paquetes enviados por el usuario.
\section{Explotación y obtención de acceso a sistemas y redes}
Existen distintas formas con las que un atacante obtiene acceso al sistema de su víctima para así obtener privilegios de instalación, ejecución de software, modificación, creación de archivos, de las que podemos destaca:
\subsection{Robo de identidad}
Está técnica usa phising para obtener datos de forma directa del usuario a traves de spam, páginas y mensajes engañosos, con el fin de que el usuario entregue sin saber datos que puedan y serán usados para atacarle.
\subsection{Engaño de firewalls e IDS's}
La manera más sencilla para engañar un firewall es la creación de "túneles", que consiste en encapsular un protocolo de red dentro de otro, con esto se pretende superar los parámetros del firewall.
\subsection{Vulnerabilidades de software}
Una vulnerabilidad de software se puede ver como un fallo en algún programa o sistema que puede ser utilizado por un virus o por un intruso, quien tendrá acceso al sistema.
\\Existen dos tipos:
\begin{itemize}
	\item[•] Buffer overflows: Esto se da gracias a algunos lenguajes de bajo nivel que permiten una mezcla entre datos introducidos por el usuario y la información del control de flujo.
	\item[•] Heap Bufferoverflow: Es un caso especial del primero, donde se utiliza el heap, y su facilidad para ser sore-escrito.
	\item[•] Vulnerabilidad de formato de cadena: Se da cuando no se válida información enviada por el usuario mediante alguna entrada.
	\item[•] SQL injection: Es la posibilidad  de  inyectar sentencias SQL (Structured Query Languaje) arbitrarias en, por ejemplo, formularios estándar publicados en un sitio web. 
\end{itemize}

\subsection{Ataques a contraseñas}
Este tipo de ataque se centra en obtener alguna contraseña para obtener acceso a mutlitud de servicios.

\subsection{Ataques a redes inalámbricas}

\subsection{Detección de intrusos}
Como hemos visto, los ataques son algo que se puede dar de multiples formas, y con ello aparte de protocolos, firewalls, precacusión, se requieren metodos para detectar si es que el equipo tiene un intruso.\\
Para ello se tiene dos tipos principales de auditorias:
\subsubsection{Registros de auditoría específicos para detección}
Podemos implementar dichos registros para que sólo nos muestre la información requerida por el sistema de detección de intrusión.
\subsubsection{Registros nativos de auditoría}
Es la herramienta que viene por defecto en los sistemas operativos y almacena toda la actividad de los usuarios, por ejemplo, el visor de eventos de Microsoft Windows.
\subsection{Detección de anomalías}
Gracias a ciertas anomalías en el perfil de un usuario de un sistema podemos detectar alguna intrusión:
\begin{itemize}
	\item[•] Contador: Es un valor que puede incrementarse mas no disminuirse hasta que sea iniciado por alguna acción
	\item[•] Calibre: Es un número que puede aumentar o disminuir, y mide el valor actual de una entidad
	\item[•] Intervalo de tiempo: Hace referencia al periodo de tiempo entre dos acontecimientos
	\item[•] Uso de recursos: Implica la cantidad de recursos que se consumen en un determinado tiempo
\end{itemize}
Para una eficiente detección se recomienda una combinación entre análisis de comportamiento y auditorias.
   
\end{document}